\documentclass{article}

\usepackage[a4paper, landscape, margin=1cm]{geometry}
\usepackage{fontspec}
\usepackage[dvipsnames]{xcolor}
\usepackage[french]{babel}
\usepackage[fontsize=6.5pt]{scrextend}
\usepackage[T1]{fontenc}
\usepackage{multicol}
\usepackage{tabularx,ragged2e}
\usepackage{sectsty}
\usepackage{lmodern}
\usepackage{stix}
\usepackage{siunitx}
\usepackage{listings}
\usepackage{multirow}
\usepackage{float}
\usepackage{titlesec}
\usepackage{fancyvrb,cprotect}
\usepackage{graphicx}
\usepackage{fontspec}
\usepackage{booktabs}
\usepackage{amsmath}
\usepackage{accents}
\usepackage{amssymb}
\usepackage{esint}
\usepackage{fancyvrb}
\usepackage{tikz}
\usepackage{enumitem}
\DefineVerbatimEnvironment{Code}{BVerbatim}{baseline=t}

\definecolor{Cornflower Blue}{RGB}{97, 149, 237}
\definecolor{RAL2000}{HTML}{da6e00}
\colorlet{corn}{Cornflower Blue}
\colorlet{heig}{RAL2000}

\input{revision}

\ifdefined\tinted
  \pagecolor{CadetBlue!60}
\fi

\newcommand*\circled[1]{\tikz[baseline=(char.base)]{
            \node[shape=circle,draw,inner sep=2pt] (char) {#1};}}
\let\code\lstinline

\setlength{\parskip}{0.2em}
\setlength{\parindent}{0em}

% Highlight configuration for C programming language
% \lstset{
%   language=tex,
%   breaklines=true,
%   keywordstyle=\bfseries\color{black},
%   basicstyle=\ttfamily\color{black},
%   emphstyle={\em \color{gray}},
%   commentstyle=\color{gray}\ttfamily,
%   mathescape=true,
%   keepspaces=true,
%   showspaces=false,
%   showtabs=true,
%   tabsize=3,
%   columns=fullflexible,
%   aboveskip=0pt,
%   belowskip=0pt
% }
\definecolor{listinggray}{gray}{0.9}
\definecolor{lbcolor}{gray}{0.92}
\lstset{
    backgroundcolor=\color{lbcolor},
    basicstyle=\tt,
    tabsize=2,
    language={[LaTeX]TeX},
    %upquote=true,
    aboveskip={0.4\baselineskip},
    belowskip={0.4\baselineskip},
    abovecaptionskip={\baselineskip},
    belowcaptionskip={0\baselineskip},
    columns=fixed,
    showstringspaces=false,
    extendedchars=true,
    %linewidth=6.7cm,
    %xleftmargin={3pt},
    %framexleftmargin={10pt},
    %framexrightmargin={2pt},
    %framextopmargin={9pt},
    %framexbottommargin={9pt},
    breaklines=true,
    %prebreak = \raisebox{0ex}[0ex][0ex]{\ensuremath{\hookleftarrow}},
    %frame=single,
    showtabs=false,
    showspaces=false,
    showstringspaces=false,
    identifierstyle=\ttfamily,
    %tagstyle=\bf,
    keywordstyle=\color{MidnightBlue},
    commentstyle=\color{ForestGreen},
    stringstyle=\color[rgb]{0.8,  0.1,  0.1},
}
% Configuration
\renewcommand{\familydefault}{\sfdefault}

\allsectionsfont{\sffamily}

% No pages numbering
\pagenumbering{gobble}

\newlength\mybaselinestretch
\mybaselinestretch=0pt plus 0.02pt\relax
\addtolength{\baselineskip}{\mybaselinestretch}

\setlength\parindent{0pt}
\setlength\tabcolsep{1.5pt}
\setlength{\columnseprule}{0.4pt}

% Macros
\newcommand{\tab}{\hspace{2em}}
\newcommand{\etc}{\small \ldots}
\newcommand{\any}{$\hzigzag$~}
\newcommand{\spc}{$\mathvisiblespace$}
\newcommand{\cd}{\lstinline}

% Titles and paragraphs more compact
% \titlespacing*{\section}{0pt}{1em}{0pt}
% \titlespacing*{\subsection}{0pt}{1em}{0pt}


\begin{document}

\begin{multicols*}{3}
\begin{tabularx}{\columnwidth}{lX}
    \raisebox{-\totalheight}{\includegraphics[width=1cm]{assets/heig-vd-black.pdf}} &
\begin{center}
  {\Large \bf Carte de référence \LaTeX} \\
  version \revision \ -- \revisiondate \\
\end{center}
\end{tabularx}
{
\scriptsize
Cette carte de référence est destinée aux étudiants utilisant \LaTeX2e pour leurs rapports de travaux pratiques ou rapports de travail de Bachelor. Cette carte comprend une liste non exhaustive des possibilités du langage \LaTeX~ résumé en une feuille A4 recto-verso. Ce document est en partie inspiré de \emph{Learn X in Y}. Il est également destiné aux étudiants qui souhaitent apprendre \LaTeX2e dans le cadre du cours de HES d'été.
}

\section*{Classes de document}
Vous pouvez également include votre propre classe définie dans un fichier \code{.cls}.
\begin{lstlisting}
\documentclass[11pt,a4paper,landscape,twoside]{article}
\documentclass{report} % Dispose de \chapter
\documentclass{book}   % Dispose de \frontmatter \mainmatter
                       % et \backmatter
\documentclass{memoir} % Pour écrire une thèse de doctorat
\documentclass{beamer} % Pour faire des présentation type PPT
\documentclass{exam}   % Pour écrire des travaux écrits
\documentclass{foobar} % Utilise la définition foobar.cls
\end{lstlisting}

\section*{Paquets}

\begin{lstlisting}
\usepackage{caption}
\usepackage{amsmath}
\usepackage{amsfonts}
\usepackage{float}
\usepackage{graphicx}
\usepackage{longtable}
\usepackage{booktabs}
\usepackage{tabularx}
\usepackage{listings}
\usepackage{xcolor}
\usepackage{xifthen}
\usepackage{hyperref}
\usepackage{fancyhdr}
\usepackage{bookmark}
\usepackage{geometry}
\usepackage{setspace}
\usepackage{babel}
\end{lstlisting}

\section*{Formats}
\begin{lstlisting}

\chapter[Description courte]{Description longue}
\section{Section 1}
\subsection{Section 1.1}
\subsubsection{Section 1.1.1}
\paragraph{Section 1.1.1.1}
\label{subsec:foobar}
Comme vu au paragraphe \ref{subsec:foobar}

\end{lstlisting}


\begin{tabularx}{\columnwidth}{lX}
\lstinline{\\textbf\{Texte en gras\}} & \textbf{Texte en gras} \\
\lstinline{\\textit\{Texte en italique\}} & \textit{Texte en italique} \\
\lstinline{\\texttt\{Texte en imprimerie\}} & \texttt{Texte en imprimerie} \\
\lstinline{\\textsc\{Texte en capitales\}} & \textsc{Texte en capitales} \\
\lstinline{\\textsf\{Texte sans-serif\}} & \textsf{Texte sans-serif} \\
\lstinline{\\emph\{Emphase\}} & \emph{Emphase} \\
\lstinline{\\fbox\{Encadré\}} & \fbox{Encadré} \\
\end{tabularx}
\section*{Taille de polices}

\begin{multicols*}{2}
  \begin{tabularx}{\columnwidth}{lX}
    \lstinline{\\tiny\{footnotesize\}} & \tiny{Tiny} \\
    \lstinline{\\scriptsize\{Script Size\}} & \scriptsize{Script Size} \\
    \lstinline{\\footnotesize\{footnotesize\}} & \footnotesize{footnotesize} \\
    \lstinline{\\small\{Small\}} & \small{Small} \\
    \lstinline{\\normalsize\{Normal Size\}} & \normalsize{Normal Size} \\
    \end{tabularx}
  \columnbreak
  \begin{tabularx}{\columnwidth}{lX}
    \lstinline{\\large\{Large\}} & \large{Large} \\
    \lstinline{\\Large\{Large\}} & \Large{Large} \\
    \lstinline{\\LARGE\{LARGE\}} & \LARGE{LARGE} \\
    \lstinline{\\huge\{Huge\}} & \huge{Huge} \\
    \lstinline{\\Huge\{Huge\}} & \Huge{Huge} \\
  \end{tabularx}
\end{multicols*}

\section*{Justification}
\begin{tabular}{*3{>{}l}}
Environnements & Déclarations & Autre \\
\code!\begin{center}! & \code!\centering! & \code!text \vfill text! \\
\code!\begin{flushleft}! & \code!\raggedright! & \code!text \hfill text! \\
\code!\begin{flushright}! & \code!\raggedleft! & \\
\end{tabular}

\section*{Minimal}

\begin{lstlisting}
\documentclass[a4paper,11pt]{article}
\usepackage[french]{babel}

\author{Yves Chevallier}
\date{\today}
\title{Apprendre \LaTeX{} avec plaisir}
\begin{document}
\maketitle
\newpage
\tableofcontents
\section{Introduction}
Bonjour.
\end{lstlisting}

\section*{Math}

\begin{tabularx}{\columnwidth}{Xl}
    \lstinline{a^2 + b^2 = c^2} & $a^2 + b^2 = c^2$ \\
\end{tabularx}

\begin{multicols*}{2}

    \begin{tabularx}{\columnwidth}{lX}
        $\alpha A$ & \lstinline{\\alpha A} \\
        $\beta B$ & \lstinline{\\beta B} \\
        $\gamma \Gamma$ & \lstinline{\\gamma \\Gamma} \\
        $\delta \Delta$ & \lstinline{\\delta \\Delta} \\
        $\epsilon \varepsilon E$ & \lstinline{\\epsilon \\varepsilon E} \\
        $\zeta Z Z$ & \lstinline{\\zeta Z} \\
        $\eta H$ & \lstinline{\\eta H} \\
        $\theta \vartheta \Theta$ & \lstinline{\\theta \\vartheta \\Theta} \\
        $\iota I$ & \lstinline{\\iota I} \\
        $\kappa K$ & \lstinline{\\kappa K} \\
        $\lambda \Lambda$ & \lstinline{\\lambda \\Lambda} \\
        $\mu M$ & \lstinline{\\mu M} \\
        $\nu N$ & \lstinline{\\nu N} \\
        $\xi \Xi$ & \lstinline{\\xi \\Xi} \\
        $o O$ & \lstinline{o O} \\
        $\pi \Pi$ & \lstinline{\\pi \\Pi} \\
        $\rho \varrho P$ & \lstinline{\\rho \\varrho P} \\
        $\sigma \Sigma$ & \lstinline{\\sigma \\Sigma} \\
        $\tau T$ & \lstinline{\\tau T} \\
        $\upsilon \Upsilon$ & \lstinline{\\upsilon \\Upsilon} \\
        $\phi \varphi \Phi$ & \lstinline{\\phi \\varphi \\Phi} \\
        $\chi X$ & \lstinline{\\chi X} \\
    $\psi \Psi$ & \lstinline{\\psi \\Psi} \\
    $\omega \Omega$ & \lstinline{\\omega \\Omega} \\
\end{tabularx}
\columnbreak
\begin{tabularx}{\columnwidth}{lX}
    $\leftarrow$ & \lstinline{\\leftarrow} \\
    $\rightarrow$ & \lstinline{\\rightarrow} \\
    $\Leftarrow$ & \lstinline{\\Leftarrow} \\
    $\Rightarrow$ & \lstinline{\\Rightarrow} \\
    $\infty$ & \lstinline{\\infty} \\
    $\forall$ & \lstinline{\\forall} \\
    $\Re$ & \lstinline{\\Re} \\
    $\Im$ & \lstinline{\\Im} \\
    $\nabla$ & \lstinline{\\nabla} \\
    $\exists$ & \lstinline{\\exists} \\
    $\nexists$ & \lstinline{\\nexists} \\
    $\partial$ & \lstinline{\\partial} \\
    $\triangle$ & \lstinline{\\triangle} \\
    $\blacksquare$ & \lstinline{\\blacksquare} \\
    $\times$ & \lstinline{\\times} \\
    $\cdot$ & \lstinline{\\cdot} \\
    $\leq$ & \lstinline{\\leq} \\
    $\geq$ & \lstinline{\\geq} \\
    $\in$ & \lstinline{\\in} \\
    $\notin$ & \lstinline{\\notin} \\
    $\vee$ & \lstinline{\\vee} \\
\end{tabularx}

\end{multicols*}
\subsection*{Fonctions importantes}
\begin{tabular}{*8{>{}l|}}
	$\sum$ & \code?\sum? & $\prod$ & \code?\prod? & $\int$ & \code?\int?\\
	$\int$ & \code?\int? & $\iint$ & \code?\iint? & $\iiint$ & \code?\iiint? & $\oint$ & \code?\oint?\\
	$\vec a$ & \code?\vec a? & $\dot a$ & \code?\dot a? & $\ddot a$ & \code?\ddot a? & $\hat a$ & \code?\hat a? \\
	\end{tabular}

\subsection*{Fonctions usuelles}
\begin{tabular}{*6{>{}l}}
	\code?\sin? & \code?\sinh? & \code?\arcsin? & \code?\csc? & \code?\ln? & \code?\min?\\
	\code?\cos? & \code?\cosh? & \code?\arccos? & \code?\sec? & \code?\lg? & \code?\max?\\
	\code?\tan? & \code?\tanh? & \code?\arctan? & \code?\cot? & \code?\log? & \code?\lim?\\
	\code?\exp? & \code?\det? & \code?\tr? & \code?\dim? & \code?\ker? & \code?\Pr?\\
	\end{tabular}

Parenthèses
\begin{multicols*}{2}
\begin{tabularx}{\columnwidth}{lX}
  $()$ & \verb+()+ \\
  $[]$ & \verb+[]+ \\
  ${}$ & \verb+\{\}+ \\
  $\langle\rangle$ & \verb+\langle\rangle+ \\
  $\lbrace\rbrace$ & \verb+\lbrace\rbrace+ \\
\end{tabularx}
\columnbreak
\begin{tabularx}{\columnwidth}{lX}
  $\lbrack\rbrack$ & \verb+\lbrack\rbrack+ \\
  $\lceil\rceil$ & \verb+\lceil\rceil+ \\
  $\lfloor\rfloor$ & \verb+\lfloor\rfloor+ \\
  $\lvert\rvert$ & \verb+\lvert\rvert+ \\
\end{tabularx}
\end{multicols*}
\section*{Équations}

\begin{lstlisting}
\begin{equation}
    c^2 = a^2 + b^2.
    \label{eq:pythagoras} % for referencing
\end{equation}
\begin{equation}
  \int_{0}^{\infty} \mathrm{e}^{-x} \mathrm{d}x
\end{equation}

\end{lstlisting}
\hrule
\begin{tabularx}{\columnwidth}{lX}
  $x^y$ &  \code+$x^y$+ \\
  $x_y$ &  \code+$x_y$+ \\
  $x^{y+z}$ &  \code+$x^{y+z}$+ \\
  $\frac{x}{y}$ & \code+$\\frac{x}{y}$+ \\
  $\sqrt{x}$ & \code+$\\sqrt{x}$+ \\
  $\sqrt[n]{x}$ & \code+$\\sqrt[n]{x}$+ \\
  % $x\cdot y$ & \code+$x\cdot y$+ \\
  % $\sum_{j=0}^n j^2$ & \code+$\sum_{j=0}^n j^2$+ \\
  % $\int_{x=0}^{\infty} \mathrm{e}^{-x} \mathrm{d}x$ & \code+$\int_{x=0}^{\infty} \mathrm{e}^{-x} \mathrm{d}x$+ \\
  % $\left\langle i,2^{2^i}\right\rangle$ & \code+$\left\langle i,2^{2^i}\right\rangle$+ \\
  % $\nabla=\boldsymbol{i}\frac{d}{dx}+...$ & \code+$\nabla=\boldsymbol{i}\frac{d}{dx}+...$+ \\
  % $\lim_{h\to 0}\frac{f(x+h-f(x)}{h}$ & \code+$\lim_{h\to 0}\frac{f(x+h-f(x)}{h}$+ \\
  % $f\colon\mathbb{R}\to\mathbb{R}$ & \code+$f\colon\mathbb{R}\to\mathbb{R}$+ \\
\end{tabularx}

Système d'équation

\begin{multicols*}{2}
\[
  f_n = \begin{cases}
    a & \text{if $n=0$} \\
    r\cdot f_n{n-1} &\text{else}
  \end{cases}
\]
\columnbreak
\begin{verbatim}
f_n = \begin{cases}
  a & \text{if $n=0$} \\
  r\cdot f_n{n-1} &\text{else}
\end{cases}
\end{verbatim}
\end{multicols*}

\begin{multicols*}{2}
  \[
    \begin{pmatrix}
      a & b \\
      c & d
    \end{pmatrix}
  \]
  \columnbreak
  \begin{verbatim}
\begin{pmatrix}
  a & b \\
  c & d
\end{pmatrix}
  \end{verbatim}
  \end{multicols*}

\section*{Figures}

\begin{multicols*}{2}


\begin{lstlisting}
\begin{figure}[H]
  \centering
  \includegraphics[width=3cm]{example-image-duck}
  \caption[Canard]{Un canard}
  \label{fig:right-triangle}
\end{figure}
\end{lstlisting}

\begin{figure}[H]
    \centering
    \includegraphics[width=3cm]{example-image-duck}
    \caption[Canard]{Un canard}
    \label{fig:right-triangle}
\end{figure}

\end{multicols*}

\section*{Table}
\subsection*{Tableau simple avec légende}
\begin{multicols*}{2}
\begin{lstlisting}
\begin{table}[H]
  \centering
  \caption{Description}
  \begin{tabular}{c|lr}
    \toprule
    Id & Nom & Prénom \\
    \midrule
    1 & Dantès & Edmond \\
    2 & Herrera & Mercédès \\
    3 & de Villefort & Gérard \\
    \bottomrule
  \end{tabular}
\end{table}
\end{lstlisting}
\columnbreak
\begin{table}[H]
  \centering
  \caption{Description}
  \begin{tabular}{c|lr}
    \toprule
    Id &  Nom & Prénom \\
    \midrule
    1 & Dantès & Edmond \\
    2 & Herrera & Mercédès \\
    3 & de Villefort & Gérard \\
    \bottomrule
  \end{tabular}
\end{table}
\end{multicols*}
\subsection*{Tableau pleine largeur avec colonne variable}
\begin{multicols*}{2}
\begin{lstlisting}
\begin{tabularx}{\textwidth}
{|c|l|X|} \hline
U & X & Y \\ \hline
1 & a & b \\ \hline
2 & c & d \\ \hline
3 & e & f \\ \hline
\end{tabularx}
\end{lstlisting}
\columnbreak
\null \vskip 0.5em
\begin{tabularx}{\columnwidth}{|c|l|X|}
  \hline
  U & X & Y \\ \hline
  1 & a & b \\
  2 & c & d \\
  3 & e & f \\ \hline
\end{tabularx}
\end{multicols*}

\subsection*{Tableau pleine largeur réparti}
\begin{multicols*}{2}
\begin{lstlisting}
\begin{tabularx}{\textwidth}
{p{5mm}|*4{>{}X|}}
1 & a & bb & ccc & dddd \\
2 & e & ff & ggg & hhhh \\
\end{tabularx}
\end{lstlisting}
\columnbreak
\null \vskip 0.5em
\begin{tabularx}{\columnwidth}{p{5mm}|*4{>{}X|}}
  1 & a & bb & ccc & dddd \\
  2 & e & ff & ggg & hhhh \\
\end{tabularx}
\end{multicols*}

\section*{Unités}
Mode d'affichage configurable avec \code+\sisetup{per-mode=reciprocal}+ ou \code+\sisetup{per-mode=fraction}+. Les préfix d'unités tels que : \code+\kilo+, \code+\mega+, \code+\giga+, \code+\tera+, \code+\peta+, \code+\micro+\dots, peuvent être utilisés devant chaque unité.
\begin{tabular}{ll}
  \num{7.123456e12} & \code+\num{7.123456e12}+ \\[1em]
	$[g] = \si{\meter \per \second \squared}$ & \code#[g] = \si{\meter\per\second\squared}#\\[1em]
	$E = \SI[per-mode=fraction]{3.7}{\kilo\micro\volt\per\milli\meter}$ & \code#E = \SI{1.3}{\kilo\volt\per\milli\meter}#\\
\end{tabular}

\section*{Couleurs}
Par défaut le package \emph{xcolor} supporte 68 couleurs standards (\textcolor{Apricot}{Apricot}, \textcolor{Bittersweet}{Bittersweet}, \textcolor{Rhodamine}{Rhodamine}, \textcolor{SpringGreen}{SpringGreen}, \dots). D'autres peuvent être définies par l'utilisateur.\par
\begin{lstlisting}
\usepackage{xcolor}
\definecolor{Cornflower Blue}{RGB}{97, 149, 237}
\definecolor{RAL2000}{HTML}{da6e00}
\colorlet{corn}{Cornflower Blue}
\colorlet{heig}{RAL2000}
\end{lstlisting}

\begin{tabularx}{\columnwidth}{lX}
  \textcolor{red}{rouge} & \verb+\textcolor{red}{rouge}+ \\
  \textcolor{corn}{Cornflower Blue} & \verb+\textcolor{corn}{Cornflower Blue}+ \\
  \textcolor{JungleGreen}{JungleGreen} & \verb+\textcolor{JungleGreen}{JungleGreen}+ \\
  \textcolor{heig}{HEIG-VD} & \verb+\textcolor{heig}{HEIG-VD}+ \\

\end{tabularx}

\section*{Mise en page}
Options de \emph{geometry}: \emph{a4paper}, \emph{a4paper, landscape}, \emph{a4paper, tight}, \emph{twocolumn}\dots
\begin{lstlisting}
\usepackage[options]{geometry}
\end{lstlisting}

\begin{multicols*}{2}
\includegraphics[height=6cm]{assets/layout.pdf}
\columnbreak
\begin{enumerate}[label=\protect\circled{\arabic*}]
  \item 1in + \code?\hoffset?
  \item 1in + \code?\voffset?
  \item \code?\oddsidemargin?
  \item \code?\topmargin?
  \item \code?\headheight?
  \item \code?\headsep?
  \item \code?\textheight?
  \item \code?\textwidth?
  \item \code?\marginparwidth?
  \item \code?\marginparwidth?
  \item \code?\footskip?
  \end{enumerate}

\end{multicols*}

\section*{Tiks}

\begin{tikzpicture}
  \draw[thick,rounded corners=8pt] (0,0) -- (0,2) -- (1,3.25)
   -- (2,2) -- (2,0) -- (0,2) -- (2,2) -- (0,0) -- (2,0);
\end{tikzpicture}

\begin{tikzpicture}[domain=0:4]
  \draw[very thin,color=gray] (-0.1,-1.1) grid (3.9,3.9);
  \draw[->] (-0.2,0) -- (4.2,0) node[right] {$x$};
  \draw[->] (0,-1.2) -- (0,4.2) node[above] {$f(x)$};
  \draw[color=red]    plot (\x,\x)             node[right] {$f(x) =x$};
  \draw[color=blue]   plot (\x,{sin(\x r)})    node[right] {$f(x) = \sin x$};
  \draw[color=orange] plot (\x,{0.05*exp(\x)}) node[right] {$f(x) = \frac{1}{20} \mathrm e^x$};
\end{tikzpicture}

\section*{Avancé}
\begin{lstlisting}
\setlength{parameters}{length}
\usecounter{counter name}
\newcounter{counter name}

\end{lstlisting}
\end{multicols*}

\end{document}
