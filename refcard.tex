\documentclass{article}

\usepackage[a4paper, landscape, margin=0.5cm]{geometry}
\usepackage{fontspec}
\usepackage[dvipsnames]{xcolor}
\usepackage[french]{babel}
\usepackage[fontsize=6.5pt]{scrextend}
\usepackage{multicol}
\usepackage{tabularx,ragged2e}
\usepackage{sectsty}
\usepackage{lmodern}
\usepackage{stix}
\usepackage{siunitx}
\usepackage{listings}
\usepackage{multirow}
\usepackage{float}
\usepackage{titlesec}
\usepackage{fancyvrb,cprotect}
\usepackage{graphicx}
\usepackage{wrapfig}
\usepackage{booktabs}
\usepackage{amsmath}
\usepackage{accents}
\usepackage{amssymb}
\usepackage{esint}
\usepackage{tikz}
\usepackage{enumitem}
\usepackage{vwcol}
\usepackage{microtype}
\usepackage[newfloat]{minted}
\usetikzlibrary{mindmap}
\DefineVerbatimEnvironment{Code}{BVerbatim}{baseline=t}

\definecolor{Cornflower Blue}{RGB}{97, 149, 237}
\definecolor{RAL2000}{HTML}{da6e00}
\colorlet{corn}{Cornflower Blue}
\colorlet{heig}{RAL2000}

\input{revision}

\ifdefined\tinted
  \definecolor{colorpage}{HTML}{cead82}
  \pagecolor{colorpage}
\fi

\newenvironment{tighttabbing}
  {\begingroup\setlength{\parskip}{0pt}\begin{tabbing}}
  {\end{tabbing}\endgroup}


\newcommand*\circled[1]{\tikz[baseline=(char.base)]{
            \node[shape=circle,draw,inner sep=2pt] (char) {#1};}}
\let\code\lstinline

\setmonofont{Source Code Pro}

% Configuration
\renewcommand{\familydefault}{\sfdefault}

% Tailles des titres (s'applique aussi aux versions étoilées *)
\titleformat{\section}
  {\sffamily\bfseries\fontsize{12}{15}\selectfont} % ajoute  si tu veux en gras
  {}{0pt}{}

\titleformat{\subsection}
  {\sffamily\bfseries\fontsize{9}{11}\selectfont}
  {}{0pt}{}

\titleformat{\subsubsection}
  {\sffamily\bfseries\fontsize{8}{10}\selectfont}
  {}{0pt}{}

% Espacements (les tiens, inchangés)
\titlespacing*{\section}{0pt}{0ex plus .1ex minus 0.5ex}{0.5ex plus .1ex minus 0.2ex}
\titlespacing*{\subsection}{0pt}{0.1ex plus .1ex minus 0.5ex}{0.5ex plus .1ex minus 0.2ex}
\titlespacing*{\subsubsection}{0pt}{0.1ex plus .1ex minus 0.5ex}{0.5ex plus .1ex minus 0.2ex}

\pagenumbering{gobble} % pas de numéros de page


\usemintedstyle{bw}

\newminted{latex}{
  fontsize=\footnotesize,
  breaklines=true,
  autogobble=true,
  linenos=false,        % passe à true si tu veux des numéros de ligne
  tabsize=3,
  mathescape,
}
% Configuration
\renewcommand{\familydefault}{\sfdefault}

\allsectionsfont{\sffamily}

\newlength\mybaselinestretch
\mybaselinestretch=0pt plus 0.02pt\relax
\addtolength{\baselineskip}{\mybaselinestretch}

\setlength\parindent{0pt}
\setlength{\parskip}{0.0em}
\setlength\tabcolsep{1.5pt}
\setlength{\columnseprule}{0.4pt}

% Titles and paragraphs more compact
\titlespacing*{\section}{0pt}{0em}{0pt}
\titlespacing*{\subsection}{0pt}{0em}{0pt}

\newcommand{\cd}[1]{\mintinline{cpp}{#1}}

\begin{document}
\raggedcolumns
\begin{multicols*}{3}
\raggedbottom
\begin{tabularx}{\columnwidth}{lX}
    \raisebox{-\totalheight}{\includegraphics[width=1cm]{assets/heig-vd-black.pdf}} &
    \begin{center}
        {\Large \bf Carte de référence \LaTeX} \\
        version \revision \ -- \revisiondate \\
    \end{center}
\end{tabularx}
{
Cette carte de référence est destinée aux étudiants utilisant \LaTeX2e pour leurs rapports de travaux pratiques ou rapports de travail de Bachelor. Cette carte comprend une liste non exhaustive des possibilités du langage \LaTeX~ résumé en une feuille A4 recto-verso. Ce document est en partie inspiré de \emph{Learn X in Y}. Il est également destiné aux étudiants qui souhaitent apprendre \LaTeX2e dans le cadre du cours de HES d'été.
}

\section*{Classes de document}
Vous pouvez également inclure votre propre classe définie dans un fichier \code{.cls}.
\begin{latexcode}
    \documentclass[11pt,a4paper,landscape,twoside]{article}
    \documentclass{report} % Dispose de \chapter
    \documentclass{book}   % Dispose de \frontmatter \mainmatter et \backmatter
    \documentclass{memoir} % Pour écrire une thèse de doctorat
    \documentclass{beamer} % Pour faire des présentation type PPT
    \documentclass{exam}   % Pour écrire des travaux écrits
    \documentclass{foobar} % Utilise la définition foobar.cls
\end{latexcode}

\section*{Quelques Paquets}

\begin{latexcode}
    \usepackage{caption}    % Configure les légendes
    \usepackage{amsmath}    % Configure l'affichage des équations
    \usepackage{float}      % Fixer les tables et figures
    \usepackage{graphicx}   % Inclusion d'images pdf, png...
    \usepackage{longtable}  % Tableaux sur plusieurs pages
    \usepackage{booktabs}   % Meilleure présentation des tableaux
    \usepackage{tabularx}   % Tableaux avec colonnes flexibles
    \usepackage{listings}   % Inclusion de code avec syntaxe
    \usepackage{xcolor}     % Usage des couleurs
    \usepackage{hyperref}   % Hyperliens pour documents PDF
    \usepackage{fancyhdr}   % En-tête et pieds de pages
    \usepackage{geometry}   % Configuration des marges
    \usepackage{babel}      % Conventions linguistiques
    \usepackage{siunitx}    % Présentation des unités physiques
    \usepackage{tikz}       % Dessin vectoriel dans LaTeX
    \usepackage{wrapfig}    % Figures avec habillage de texte
    \usepackage{subcaption} % Sous-figures et sous-tables
    \usepackage{multicol}   % Colonnes multiples

\end{latexcode}

\section*{Formats et références}
\begin{latexcode}
    \tableofcontents   % Table des matières
    \listoffigures     % Liste des figures
    \listoftables      % Liste des tableaux
    \listoflistings    % Liste des listes de code
    \index             % Index des termes
    \printindex        % Affichage de l'index
    \printglossary     % Affichage du glossaire
    \printbibliography % Affichage de la bibliographie

    \chapter[Description courte]{Description longue}
    \label{subsec:foobar}
    \section{Section 1}
    \subsection{Section 1.1}
    \subsubsection{Section 1.1.1}
    \paragraph{Section 1.1.1.1}
    \subsection*{Section sans numéro}
    Comme vu au paragraphe \ref{subsec:foobar}
\end{latexcode}

\section*{Commandes pratiques}
\begin{tabular}{*2{>{}l}}
    \code!\today!           & Date du jour                           \\
    \code!\LaTeX!           & Logo \LaTeX                            \\
    \code!\TeX!             & Logo \TeX                              \\
    \code!\newpage!         & Nouvelle page                          \\
    \code!\clearpage!       & Nouvelle page + vider les flottants    \\
    \code!\pagebreak!       & Saut de page                           \\
    \code!\linebreak!       & Saut de ligne                          \\
    \code!\noindent!        & Pas d'indentation                      \\
    \code!\phantom{texte}!  & Espace invisible de la taille du texte \\
    \code!\hphantom{texte}! & Espace horizontal invisible            \\
    \code!\vphantom{texte}! & Espace vertical invisible              \\
    \code!\protect!         & Protéger une commande fragile          \\
\end{tabular}

\section*{Polices et styles}
\subsection*{Format de police}
\begin{tabularx}{\columnwidth}{lX}
    \lstinline{\\textbf\{Texte en gras\}}       & \textbf{Texte en gras}       \\
    \lstinline{\\textit\{Texte en italique\}}   & \textit{Texte en italique}   \\
    \lstinline{\\texttt\{Texte en imprimerie\}} & \texttt{Texte en imprimerie} \\
    \lstinline{\\textsc\{Texte en capitales\}}  & \textsc{Texte en capitales}  \\
    \lstinline{\\textsf\{Texte sans-serif\}}    & \textsf{Texte sans-serif}    \\
    \lstinline{\\emph\{Emphase\}}               & \emph{Emphase}               \\
    \lstinline{\\fbox\{Encadré\}}               & \fbox{Encadré}               \\
\end{tabularx}

\subsection*{Taille de polices}
\begin{multicols*}{2}
    \begin{tabularx}{\columnwidth}{lX}
        \lstinline{\\tiny\{Tiny\}}             & \tiny{Tiny}             \\
        \lstinline{\\scriptsize\{Script\}}     & \scriptsize{Script}     \\
        \lstinline{\\footnotesize\{Footnote\}} & \footnotesize{Footnote} \\
        \lstinline{\\small\{Small\}}           & \small{Small}           \\
        \lstinline{\\normalsize\{Normal\}}     & \normalsize{Normal}     \\
    \end{tabularx}
    \columnbreak
    \begin{tabularx}{\columnwidth}{lX}
        \lstinline{\\large\{Large\}} & \large{Large} \\
        \lstinline{\\Large\{Large\}} & \Large{Large} \\
        \lstinline{\\LARGE\{LARGE\}} & \LARGE{LARGE} \\
        \lstinline{\\huge\{Huge\}}   & \huge{Huge}   \\
        \lstinline{\\Huge\{Huge\}}   & \Huge{Huge}
    \end{tabularx}
\end{multicols*}

\subsection*{Listes}
\begin{minipage}[c]{0.3\linewidth}
    \begin{itemize}
        \item Item 1
        \item Item 2
        \item Item 3
    \end{itemize}
\end{minipage}
\begin{minipage}[c]{0.6\linewidth}
    \begin{latexcode}
        \begin{itemize}
            \item Item 1
            \item Item 2
            \item Item 3 \label{it:item3}
        \end{itemize}
    \end{latexcode}
\end{minipage}

\begin{minipage}[c]{0.3\linewidth}
    \begin{enumerate}
        \item Item 1
        \item Item 2
        \item Item 3
    \end{enumerate}
\end{minipage}
\begin{minipage}[c]{0.6\linewidth}
    \begin{latexcode}
        \begin{enumerate}
            \item Item 1
            \item Item 2
            \item Item 3 \label{it:item3}
        \end{enumerate}
    \end{latexcode}
\end{minipage}
\begin{latexcode}
    \usepackage{enumitem}
    \setlist[enumerate,1]{label=\Alph*)}   % A) B) C)
    \setlist[enumerate,2]{label=\alph*)}   % a) b) c)
    \setlist[enumerate,3]{label=\roman*)}  % i) ii) iii)
    \setlist[enumerate,4]{label=\arabic*)} % 1) 2) 3)
\end{latexcode}

\section*{Fichiers séparés}
\begin{latexcode}
    \input{file.tex}   % Inclusion d'un fichier
    \include{file.tex} % Inclusion d'un fichier et nouvelle page
    \includeonly{file1.tex,file2.tex} % Inclusion de fichiers
\end{latexcode}

\section*{Caractère spéciaux}
\begin{multicols*}{4}
\begin{tighttabbing}
    aa \= a \kill
    \S              \> \code+\S+              \\
    \textasciicircum \> \code+\textasciicircum+ \\
    \textasciitilde \> \code+\textasciitilde+ \\
    \textbackslash   \> \code+\textbackslash+   \\
    \textbar        \> \code+\textbar+        \\
    \textbraceleft   \> \code+\textbraceleft+   \\
    \textbraceright \> \code+\textbraceright+ \\
    \textbullet      \> \code+\textbullet+      \\
    \textcent       \> \code+\textcent+       \\
    \textdollar      \> \code+\textdollar+      \\
    $\ldots$        \> \code+$\ldots$+        \\
    $\widehat{oo}$   \> \code+$\widehat{oo}$+   \\
    $\clubsuit$     \> \code+$\clubsuit$+     \\
    $\spadesuit$     \> \code+$\spadesuit$+     \\
\end{tighttabbing}
\end{multicols*}

% \subsection*{Accents français}
% \begin{multicols*}{4}
% \begin{tighttabbing}
%     aa \= a \kill
%     à \> \code+\`{a}+ ou \verb+à+ \\
%     é \> \code+\'{e}+ ou \verb+é+ \\
%     è \> \code+\`{e}+ ou \verb+è+ \\
%     ê \> \code+\^{e}+ ou \verb+ê+ \\
%     ë \> \code+\"{e}+ ou \verb+ë+ \\
%     ç \> \code+\c{c}+ ou \verb+ç+ \\
%     î \> \code+\^{i}+ ou \verb+î+ \\
%     ï \> \code+\"{i}+ ou \verb+ï+ \\
%     ô \> \code+\^{o}+ ou \verb+ô+ \\
%     ù \> \code+\`{u}+ ou \verb+ù+ \\
%     û \> \code+\^{u}+ ou \verb+û+ \\
%     ü \> \code+\"{u}+ ou \verb+ü+
% \end{tighttabbing}
% \end{multicols*}

% Avec \emph{babel} français et encodage UTF-8, les accents peuvent être tapés directement.

\subsection*{Espaces et typographie}
\begin{multicols*}{2}
\begin{tighttabbing}
    aaaaaa \= a \kill
\code!\,! ou \code!\ !    \> Espace fine (avant : ; ! ?)   \\
    \code!\quad!          \> Espace large (1 em)           \\
    \code!\qquad!         \> Espace très large (2 em)      \\
    \code!~!              \> Espace insécable              \\
    \code!\@!             \> Espace normale après un point \\
    \code!\\!             \> Saut de ligne \\
    \code!\\*!            \> Saut de ligne sans saut de page \\
    \code!\-!                 \> Césure suggérée               \\
    \code!\mbox{texte}!        Empêcher la césure            \\
    \code!\hyphenation{mots}!  Définition des césures
\end{tighttabbing}
\end{multicols*}


Règles typo. françaises : espace fine insécable avant \code!: ; ! ?! et après les guillemets \og{} et \fg{}.


\subsection*{Alignement et espacements}
\begin{tabular}{*2{>{}l}}
\code!\begin{center}! & \code!\centering! \\
\code!\begin{flushleft}! & \code!\raggedright! \\
\code!\begin{flushright}! & \code!\raggedleft! \\
\code!\vspace{1cm}! & Espacement vertical \\
\code!\hspace{1cm}! & Espacement horizontal \\
\code!\vfil! & Remplissage vertical automatique \\
\code!\hfil! & Remplissage horizontal automatique \\
\code!\vfill! & Remplissage vertical jusqu'au bas de page \\
\code!\hfill! & Remplissage horizontal jusqu'au bord de page \\
\end{tabular}

\subsection*{Environnements importants}
\begin{tabular}{*2{>{}l}}
\code!\begin{quote}! & Citation courte \\
\code!\begin{quotation}! & Citation longue avec alinéas \\
\code!\begin{verse}! & Poésie, vers \\
\code!\begin{abstract}! & Résumé \\
\code!\begin{appendix}! & Section annexe \\
%\code!\begin{verbatim}! & Texte littéral (sans interprétation) \\
\code!\begin{minipage}{width}! & Page réduite dans la page \\
\end{tabular}

\section*{Unités internes et longueurs}
\begin{tabular}{ll|ll|ll|ll}
    Points ($\SI{351.46}{\mu\meter}$) & \code+pt+ & Millimètres & \code+mm+ & Pouce ($\SI{2.54}{\centi\metre}$) & \code+in+ & Largeur d'un M & \code+em+ \\
    Pixel                             & \code+px+ & Centimètres & \code+cm+ & Pica (12 Points)                  & \code+pc+ & Hauteur d'un X & \code+ex+ \\
\end{tabular}

\begin{tabular}{*2{>{}l}}
    \code!\baselineskip!  & Espacement vertical entre paragraphes            \\
    \code!\columnsep!     & Espacement horizontal entre colonnes             \\
    \code!\columnwidth!   & Largeur d'une colonne                            \\
    \code!\linewidth!     & Largeur de la ligne dans l'environnement courant \\
    \code!\textwidth!     & Largeur de la page moins les marges              \\
    \code!\parskip!       & Espacement horizontal de la première ligne       \\
    \code!\paperwidth!    & Largeur de la page                               \\
    \code!\paperheight!   & Hauteur de la page                               \\
    \code!0.25\textwidth! & \SI{25}{\percent} de la largeur de la page       \\
\end{tabular}

Définition d'une longueur: \code!\setlength{<nom>}{<valeur>}! p.ex. \\
\code!\setlength{\columnsep}{1cm}!
% \section*{Exemple minimal}

% \begin{multicols*}{2}
% \begin{latexcode}
%     \documentclass[a4paper,11pt]{article}
%     \usepackage[french]{babel}
%     \author{Yves Chevallier}
%     \date{\today}
%     \title{Apprendre \LaTeX{} avec joie}
%     \begin{document}
%     \maketitle \newpage
%     \tableofcontents
%     \section{Introduction}
%     Bonjour, monde!
%     \end{document}
% \end{latexcode}
% \end{multicols*}

\section*{Math}

\begin{tabularx}{\columnwidth}{lX}
    \lstinline{a^2 + b^2 = c^2} & $a^2 + b^2 = c^2$     \\
    \lstinline{$...$}           & Mode math en ligne    \\
    \lstinline{$$...$$}         & Mode math multi-ligne \\
    \lstinline{                                         \\[...\\]} & Mode math multi-ligne \\
    \lstinline{\\begin\{equation\}...                   \\end\{equation\}} & Équation numérotée \\
    \lstinline{\\begin\{equation*\}...                  \\end\{equation*\}} & Équation non numérotée \\
\end{tabularx}

Sur plusieurs lignes :\par
\begin{minipage}[c]{0.3\linewidth}
    \begin{multline*}
        \sin(x)=x-\frac{x^3}{3!} \\
        +\frac{x^5}{5!}-\frac{x^7}{7!}+\cdots
    \end{multline*}
\end{minipage}
\begin{minipage}[c]{0.7\linewidth}
    \begin{latexcode}
        \begin{multline*}
            \sin(x)=x-\frac{x^3}{3!} \\
            +\frac{x^5}{5!}-\frac{x^7}{7!}+\cdots
        \end{multline*}
    \end{latexcode}
\end{minipage}

Ou alignées :\par
\begin{minipage}[c]{0.3\linewidth}
    \begin{align*}
        \nabla\cdot\boldsymbol{D} & = \rho \\
        \nabla\cdot\boldsymbol{B} & = 0    \\
    \end{align*}
\end{minipage}
\begin{minipage}[c]{0.7\linewidth}
    \begin{latexcode}
        \begin{align*}
            \nabla\cdot\boldsymbol{D} & = \rho \\
            \nabla\cdot\boldsymbol{B} & = 0    \\
        \end{align*}
    \end{latexcode}
\end{minipage}

\subsection*{Symboles mathématiques}
\begin{multicols*}{3}
    \begin{tighttabbing}
        aaaaa \= a \kill
        $\alpha A$                \> \lstinline{\alpha A} \\
        $\beta B$                 \> \lstinline{\beta B} \\
        $\gamma \Gamma$           \> \lstinline{\gamma \\Gamma} \\
        $\delta \Delta$           \> \lstinline{\delta \\Delta} \\
        $\epsilon E$  \> \lstinline{\epsilon E} \\
        $\varepsilon$ \> \lstinline{\varepsilon} \\
        $\zeta Z Z$               \> \lstinline{\zeta Z} \\
        $\eta H$                  \> \lstinline{\eta H} \\
        $\theta \vartheta$ \> \lstinline{\theta \\vartheta} \\
        $\Theta$                   \> \lstinline{\Theta} \\
        $\iota I$                 \> \lstinline{\iota I} \\
        $\kappa K$                \> \lstinline{\kappa K} \\
        $\lambda \Lambda$         \> \lstinline{\lambda \\Lambda} \\
        $\mu M$                   \> \lstinline{\mu M} \\
        $\nu N$                   \> \lstinline{\nu N} \\
        $\xi \Xi$                 \> \lstinline{\xi \\Xi} \\
        $o O$                     \> \lstinline{o O} \\
        $\pi \Pi$                 \> \lstinline{\pi \\Pi} \\
        $\rho \varrho P$          \> \lstinline{\rho \\varrho P} \\
        $\sigma \Sigma$           \> \lstinline{\sigma \\Sigma} \\
        $\tau T$                  \> \lstinline{\tau T} \\
        $\upsilon \Upsilon$       \> \lstinline{\upsilon \\Upsilon} \\
        $\phi \varphi \Phi$       \> \lstinline{\phi \\varphi \\Phi} \\
        $\chi X$                  \> \lstinline{\chi X} \\
        $\psi \Psi$               \> \lstinline{\psi \\Psi} \\
        $\omega \Omega$ \> \lstinline{\omega \\Omega} \\
        $\leftarrow$    \> \lstinline{\leftarrow} \\
        $\rightarrow$   \> \lstinline{\rightarrow} \\
        $\Leftarrow$    \> \lstinline{\Leftarrow} \\
        $\Rightarrow$   \> \lstinline{\Rightarrow} \\
        $\infty$        \> \lstinline{\infty} \\
        $\forall$       \> \lstinline{\forall} \\
        $\Re$           \> \lstinline{\Re} \\
        $\Im$           \> \lstinline{\Im} \\
        $\nabla$        \> \lstinline{\nabla} \\
        $\exists$       \> \lstinline{\exists} \\
        $\nexists$      \> \lstinline{\nexists} \\
        $\partial$      \> \lstinline{\partial} \\
        $\triangle$     \> \lstinline{\triangle} \\
        $\blacksquare$  \> \lstinline{\blacksquare} \\
        $\times$        \> \lstinline{\times} \\
        $\cdot$         \> \lstinline{\cdot} \\
        $\leq$          \> \lstinline{\leq} \\
        $\geq$          \> \lstinline{\geq} \\
        $\in$           \> \lstinline{\in} \\
        $\notin$        \> \lstinline{\notin} \\
        $\vee$          \> \lstinline{\vee} \\
    \end{tighttabbing}

\end{multicols*}
\subsection*{Fonctions importantes}
\begin{multicols*}{4}
\begin{tighttabbing}
    aaa \= a \kill
    $\sum$    \> \code?\sum?    \\
    $\prod$  \> \code?\prod?  \\
    $\int$   \> \code?\int?   \\
    $\int$    \> \code?\int?    \\
    $\iint$  \> \code?\iint?  \\
    $\iiint$ \> \code?\iiint? \\
    $\oint$ \> \code?\oint? \\
    $\vec a$ \> \code?\vec a? \\
    $\dot a$ \> \code?\dot a? \\
    $\ddot a$ \> \code?\ddot a? \\
    $\hat a$ \> \code?\hat a?  \\
\end{tighttabbing}
\end{multicols*}

\subsection*{Fonctions usuelles}
\begin{multicols*}{3}
\begin{tighttabbing}
    aaaaaa \= a \kill
    \code?\sin? \> Sinus \\
    \code?\sinh? \> Sinus hyp. \\
    \code?\arcsin? \> Arcsinus \\
    \code?\csc? \> Cosécante \\
    \code?\ln? \> Logarithme népérien \\
    \code?\min? \> Minimum \\
    \code?\cos? \> Cosinus \\
    \code?\cosh? \> Cosinus hyp. \\
    \code?\arccos? \> Arccosinus \\
    \code?\sec? \> Sécante \\
    \code?\lg? \> Logarithme décimal \\
    \code?\max? \> Maximum \\
    \code?\tan? \> Tangente \\
    \code?\tanh? \> Tangente hyp. \\
    \code?\arctan? \> Arctangente \\
    \code?\cot? \> Cotangente \\
    \code?\log? \> Log. en base $n$ \\
    \code?\lim? \> Limite \\
    \code?\exp? \> Exponentielle \\
    \code?\det?  \> Déterminant \\
    \code?\tr?  \> Trace \\
    \code?\dim? \> Dimension \\
    \code?\ker? \> Noyau \\
    \code?\Pr?  \> Probabilité
\end{tighttabbing}
\end{multicols*}

\subsection*{Parenthèses et délimiteurs}
Utiliser \code+\left+ (\code+\right+) pour ajuster la hauteur des parenthèses à l'expression.
\begin{tabular}{p{0.5cm}p{2cm}p{0.5cm}p{2cm}p{0.5cm}p{2cm}}
    $(.)$             & \code+(.)+             & $[.]$             & \code+[.]+             & $\{.\}$           & \code+\{.\}+           \\
    $\langle.\rangle$ & \code+\langle.\rangle+ & $\lbrace.\rbrace$ & \code!\lbrace.\rbrace! & $\lvert.\rvert$   & \code!\lvert.\rvert!   \\
    $\lbrack.\rbrack$ & \code!\lbrack.\rbrack! & $\lceil.\rceil$   & \code!\lceil.\rceil!   & $\lfloor.\rfloor$ & \code!\lfloor.\rfloor! \\
\end{tabular}
\subsection*{Élements d'équations}

\begin{multicols*}{2}
\begin{tighttabbing}
    aaaaaa \= a \kill
    $x^y$                                             \> \code+$x^y$+                                             \\
    $x_y$                                             \> \code+$x_y$+                                             \\
    $x^{y+z}$                                         \> \code!$x^{y+z}$!                                         \\
    $\frac{x}{y}$                                     \> \code!$\frac{x}{y}$!                                     \\
    $\sqrt{x}$                                        \> \code+$\sqrt{x}$+                                        \\
    $\sqrt[n]{x}$                                     \> \code+$\sqrt[n]{x}$+                                     \\
    $x\cdot y$                                        \> \code+$x\cdot y$+                                        \\
    $\sum_{j=0}^n j^2$                                \> \code+$\sum_{j=0}^n j^2$+
\end{tighttabbing}
\end{multicols*}

\begin{tabular}{p{2cm}l}

    $\left(a + \frac{1}{2} \right)^2$                 & \code!$\left( a + \frac{1}{2} \right)^2$!                \\
    $\int_{x=0}^{\infty} \mathrm{e}^{-x} \mathrm{d}x$ & \code+$\int_{x=0}^{\infty} \mathrm{e}^{-x} \mathrm{d}x$+ \\
    $\left\langle i,2^{2^i}\right\rangle$             & \code+$\left\langle i,2^{2^i}\right\rangle$+             \\
    $\nabla=\boldsymbol{i}\frac{d}{dx}+...$           & \code!$\nabla=\boldsymbol{i}\frac{d}{dx}+...$!           \\
    $\lim_{h\to 0}\frac{f(x+h-f(x)}{h}$               & \code!$\lim_{h\to 0}\frac{f(x+h-f(x)}{h}$!               \\
    $f\colon\mathbb{R}\to\mathbb{R}$                  & \code+$f\colon\mathbb{R}\to\mathbb{R}$+                  \\
\end{tabular}

\subsection*{Système d'équation}

\begin{multicols*}{2}
    \[
        f_n = \begin{cases}
            a              & \text{if $n=0$} \\
            r\cdot f_{n-1} & \text{else}
        \end{cases}
    \]
    \columnbreak
    \begin{latexcode}
        f_n = \begin{cases}
            a              & \text{if $n=0$} \\
            r\cdot f_{n-1} & \text{else}
        \end{cases}
    \end{latexcode}
\end{multicols*}

\begin{multicols*}{2}
    \[
        \begin{pmatrix}
            a & b \\
            c & d
        \end{pmatrix}
    \]
    \columnbreak
    \begin{latexcode}
        \begin{pmatrix}
            a & b \\
            c & d
        \end{pmatrix}
    \end{latexcode}
\end{multicols*}

\section*{Figures}

\begin{multicols*}{2}
\begin{tighttabbing}
    aa \= a \kill
    h \> Placer la fig. approx. ici \\
    t \> Placer la fig. au début de la page   \\
    b \> Placer la fig. à la fin de la page   \\
    p \> Placer la fig. sur une page à part   \\
    ! \> Surcharge la position par défaut       \\
    H \> Force la fig. à être placée ici      \\
\end{tighttabbing}
\end{multicols*}

\begin{latexcode}
    \graphicspath{ {images/} } % Chemin des images
    \includegraphics[scale=1.5]{logo} % Inclure une image
    \includegraphics[width=\textwidth]{logo} % Inclure une image

    \begin{wrapfigure}{r}{0.25\textwidth}
        \centering
        \includegraphics[width=0.25\textwidth]{image}
    \end{wrapfigure} % Image encapsulée dans le texte

    % Sous-figures avec subcaption
    \begin{figure}[H]
        \centering
        \begin{subfigure}{0.45\textwidth}
            \includegraphics[width=\textwidth]{image1}
            \caption{Première image}
        \end{subfigure}
        \begin{subfigure}{0.45\textwidth}
            \includegraphics[width=\textwidth]{image2}
            \caption{Seconde image}
        \end{subfigure}
        \caption{Deux images côte à côte}
    \end{figure}
\end{latexcode}
\begin{multicols*}{2}

    \begin{latexcode}
        \begin{figure}[H]
            \centering
            \includegraphics[width=2cm]{img}
            \caption[Canard]{Un canard}
            \label{fig:right-triangle}
        \end{figure}
    \end{latexcode}

    \begin{figure}[H]
        \centering
        \includegraphics[width=2cm]{example-image-duck}
        \caption[Canard]{Un canard}
        \label{fig:right-triangle}
    \end{figure}

\end{multicols*}

\section*{Table}
\subsection*{Tableau simple avec légende}
\begin{minipage}[c]{0.7\linewidth}
    \begin{latexcode}
        \begin{table}[H] \centering
            \caption{Description \label{tb:tab1}}
            \begin{tabular}{c|lr} \toprule
                Id & Nom     & Prénom   \\ \midrule
                1  & Dantès  & Edmond   \\
                2  & Herrera & Mercédès \\ \bottomrule
            \end{tabular} \end{table}
    \end{latexcode}
\end{minipage}
\begin{minipage}[c]{0.3\linewidth}
    \begin{table}[H]
        \centering
        \caption{Description}
        \begin{tabular}{c|lr}
            \toprule
            Id & Nom     & Prénom   \\
            \midrule
            1  & Dantès  & Edmond   \\
            2  & Herrera & Mercédès \\
            \bottomrule
        \end{tabular}
    \end{table}
\end{minipage}

\subsection*{Tableau pleine largeur avec colonne variable}
\begin{multicols*}{2}
    \begin{latexcode}
        \begin{tabularx}{\textwidth}
            {|c|l|X|} \hline
            U & X & Y \\ \hline
            1 & a & b \\ \hline
            2 & c & d \\ \hline
            3 & e & f \\ \hline
        \end{tabularx}
    \end{latexcode}
    \columnbreak
    \null \vskip 0.5em
    \begin{tabularx}{\columnwidth}{|c|l|X|}
        \hline
        U & X & Y \\ \hline
        1 & a & b \\
        2 & c & d \\
        3 & e & f \\ \hline
    \end{tabularx}
\end{multicols*}

\subsection*{Tableau pleine largeur réparti}
\begin{multicols*}{2}
    \begin{latexcode}
        \begin{tabularx}{\textwidth}
            {p{5mm}|*4{>{}X|}}
            1 & a & bb & ccc & dddd \\
            2 & e & ff & ggg & hhhh \\
        \end{tabularx}
    \end{latexcode}
    \columnbreak
    \null \vskip 0.5em
    \begin{tabularx}{\columnwidth}{p{5mm}|*4{>{}X|}}
        1 & a & bb & ccc & dddd \\
        2 & e & ff & ggg & hhhh \\
    \end{tabularx}
\end{multicols*}

\section*{Unités}
Mode d'affichage configurable avec \code+\sisetup{per-mode=reciprocal}+ ou \code+\sisetup{per-mode=fraction}+. Les préfix d'unités tels que : \code+\kilo+, \code+\mega+, \code+\giga+\dots, peuvent être utilisés devant chaque unité.

\begin{tabular}{p{2cm}l}
    \num{7.123456e12}                                          & \code+\num{7.123456e12}+                                        \\
    $[g] = \si{\meter\per\second\squared}$                     & \code!$[g] = \si{\meter\per\second\squared}$!                   \\
    $E = \SI[per-mode=fraction]{3.7}{\volt\per\milli\meter}$   & \code!E = \SI{3.7}{\volt\per\milli\meter}!                      \\
    $E = \SI[per-mode=reciprocal]{3.7}{\volt\per\milli\meter}$ & \code!E = \SI[per-mode=reciprocal]{3.7}{\volt\per\milli\meter}! \\[1em]
\end{tabular}

\section*{Couleurs}
Par défaut le package \emph{xcolor} supporte 68 couleurs standards (\textcolor{Apricot}{Apricot}, \textcolor{Bittersweet}{Bittersweet}, \textcolor{Rhodamine}{Rhodamine}, \textcolor{SpringGreen}{SpringGreen}, \dots). D'autres peuvent être définies par l'utilisateur.\par
\begin{latexcode}
    \usepackage{xcolor}
    \definecolor{Cornflower Blue}{RGB}{97, 149, 237}
    \definecolor{RAL2000}{HTML}{da6e00}
    \colorlet{corn}{Cornflower Blue}
    \colorlet{heig}{RAL2000}
\end{latexcode}

\begin{multicols*}{2}
\begin{tabbing}
    aaaaaaaaa \= a \kill
    \textcolor{red}{texte}               \> \code!\textcolor{red}{texte}!              \\
    \textcolor{corn}{texte}    \> \code!\textcolor{corn}{texte}!    \\
    \textcolor{JungleGreen}{texte} \> \code!\textcolor{JungleGreen}{texte}! \\
    \textcolor{heig}{texte}            \> \code!\textcolor{heig}{texte}!            \\
    \textcolor{blue}{texte}              \> \code!\textcolor{blue}{texte}!             \\
    \colorbox{yellow}{texte}            \> \code!\colorbox{yellow}{texte}!            \\
    \fcolorbox{red}{yellow}{texte}      \> \code!\fcolorbox{red}{yellow}{texte}!
\end{tabbing}
\end{multicols*}

% \section*{Mise en page}
% Options de \emph{geometry}: \emph{a4paper}, \emph{a4paper, landscape}, \emph{a4paper, tight}, \emph{twocolumn}\dots
% \begin{latexcode}
%     \usepackage[options]{geometry}
%     \usepackage[left=2cm,right=2cm,top=2cm,bottom=2cm]{geometry}
%     \usepackage[a4paper,landscape,tight,twocolumn]{geometry}

% \end{latexcode}

% \begin{multicols*}{2}
%     %\includegraphics[height=6cm]{assets/layout.pdf}
%     \columnbreak
%     \begin{enumerate}[label=\protect\circled{\arabic*}]
%         \item 1in + \code?\hoffset?
%         \item 1in + \code?\voffset?
%         \item \code?\oddsidemargin?
%         \item \code?\topmargin?
%         \item \code?\headheight?
%         \item \code?\headsep?
%         \item \code?\textheight?
%         \item \code?\textwidth?
%         \item \code?\marginparwidth?
%         \item \code?\marginparwidth?
%         \item \code?\footskip?
%     \end{enumerate}

% \end{multicols*}


% \section*{Tiks}

% \begin{tikzpicture}
%   \draw[thick,rounded corners=8pt] (0,0) -- (0,2) -- (1,3.25)
%    -- (2,2) -- (2,0) -- (0,2) -- (2,2) -- (0,0) -- (2,0);
% \end{tikzpicture}

% \begin{tikzpicture}[domain=0:4]
%   \draw[very thin,color=gray] (-0.1,-1.1) grid (3.9,3.9);
%   \draw[->] (-0.2,0) -- (4.2,0) node[right] {$x$};
%   \draw[->] (0,-1.2) -- (0,4.2) node[above] {$f(x)$};
%   \draw[color=red]    plot (\x,\x)             node[right] {$f(x) =x$};
%   \draw[color=blue]   plot (\x,{sin(\x r)})    node[right] {$f(x) = \sin x$};
%   \draw[color=orange] plot (\x,{0.05*exp(\x)}) node[right] {$f(x) = \frac{1}{20} \mathrm e^x$};
% \end{tikzpicture}

% \begin{multicols*}{2}


% \resizebox{5cm}{!}{%
% \begin{tikzpicture}[
%   mindmap,
%   concept color = gray!30,
%   every node/.style = {concept},
%   grow cyclic,
%   level 1/.append style = {
%       concept color = gray!20,
%       level distance = 4.5cm,
%       sibling angle = 120
%   },
%   level 2/.append style = {
%       concept color = gray!10,
%       level distance = 3cm,
%       sibling angle = 45
%   }
% ]

% \node  {Our root concept}
%   child {node {First idea}
%       child {node {Fact 1}}
%       child {node {Fact 2}}
%   }
%   child {node {Second idea}
%   }
%   child {node {Third idea}
%       child {node {Note 1}}
%       child {node {Note 2}}
%       child {node {Note 3}}
% };

% \end{tikzpicture}
% }
% \columnbreak
% \begin{latexcode}
%   \begin{tikzpicture}[
%     mindmap,
%     concept color = gray!30,
%     every node/.style = {concept},
%     grow cyclic,
%     level 1/.append style = {
%         concept color = gray!20,
%         level distance = 4.5cm,
%         sibling angle = 120
%     },
%     level 2/.append style = {
%         concept color = gray!10,
%         level distance = 3cm,
%         sibling angle = 45
%     }
%   ]

%   \node  {Our root concept}
%     child {node {First idea}
%         child {node {Fact 1}}
%         child {node {Fact 2}}
%     }

%     child {node {Third idea}
%         child {node {Note 1}}
%         child {node {Note 2}}
%         child {node {Note 3}}
%   };

%   \end{tikzpicture}
% \end{latexcode}
% \end{multicols*}

% \section*{Bibliographie}

\section*{Compteurs}
\begin{latexcode}
    \setlength{parameters}{length}
    \usecounter{counter name}
    \newcounter{counter name}
\end{latexcode}

\section*{Bibliographie}
\begin{latexcode}
    % Avec bibtex (traditionnel)
    \bibliographystyle{plain} % plain, alpha, unsrt, abbrv
    \bibliography{references} % fichier references.bib

    % Avec biblatex (moderne)
    \usepackage[style=numeric,backend=biber]{biblatex}
    \addbibresource{references.bib}
    \cite{key}           % Citation simple
    \citep{key}          % Citation entre parenthèses
    \citet{key}          % Citation textuelle
    \nocite{*}           % Citer toutes les références
    \printbibliography   % Afficher la bibliographie
\end{latexcode}

\section*{Index et glossaire}
\begin{latexcode}
    % Index
    \usepackage{makeidx}
    \makeindex
    \index{terme}        % Ajouter un terme à l'index
    \index{terme!sous-terme} % Sous-entrée
    \printindex          % Afficher l'index

    % Glossaire
    \usepackage{glossaries}
    \makeglossaries
    \newglossaryentry{latex}{name=LaTeX,description={Document preparation system}}
    \gls{latex}          % Utiliser un terme du glossaire
    \printglossaries     % Afficher le glossaire
\end{latexcode}

\section*{Code source}
Deux paquets : \emph{listings} ou \emph{minted}.
\begin{multicols*}{2}
\begin{latexcode}
\usepackage{listings}
\lstset{
  backgroundcolor=\color{Red},
  basicstyle=\ttfamily\small,
  tabsize=2,
  language={C++}}
\lstinputlisting{code.cpp}
\begin{lstlisting}[
    language=C++,caption=Mon code]
int main() { return 0; }
\end{lstlisting}
\end{latexcode}
\columnbreak
\begin{latexcode}
\usepackage{minted} % --shell-escape
\begin{listing}[h]
  \inputminted{python}{code.py}
  \caption{Blabla \label{py:code}}
\end{listing}
\end{latexcode}
\end{multicols*}

% \section*{Erreurs courantes et débogage}
% \begin{tabular}{p{2.8cm}l}
% \texttt{! Missing \$ inserted} & Mode math requis: utiliser \$\dots\$ \\
% \texttt{! Undefined control sequence} & Commande inconnue ou package manquant \\
% \texttt{! LaTeX Error: File not found} & Fichier d'image ou package introuvable \\
% \texttt{! Package babel Error} & Problème d'encodage: utiliser UTF-8 \\
% \texttt{! Extra alignment tab} & Trop de colonnes dans le tableau \\
% \texttt{! Missing \} inserted} & Accolade fermante manquante \\
% \texttt{Overfull \\hbox} & Ligne trop longue, couper le mot \\
% \texttt{! LaTeX Error: \\begin\{...\} ended by \\end\{...\}} & Environnements non appariés \\
% \end{tabular}

% \begin{latexcode}
%     % Quelques astuces de débogage :
%     \listfiles              % Liste les packages utilisés
%     \fbox{texte}           % Encadrer pour voir les dimensions
%     % Compilation en deux passes pour les références
%     % Vider le cache : rm *.aux *.log *.toc *.lof *.lot

%     % Compilation typique :
%     % pdflatex document.tex    (1ère passe)
%     % bibtex document          (si bibliographie)
%     % pdflatex document.tex    (2ème passe pour références)
%     % pdflatex document.tex    (3ème passe pour finir)

%     % Ou avec latexmk (automatique) :
%     % latexmk -pdf document.tex
% \end{latexcode}

\section*{Références}
\begin{tabular}{p{1.8cm}p{2.2cm}p{2cm}l}
    \code!\cite{key}!      & Cite une référence   & \code!\label{marker}!   & Définit un label     \\
    \code!\ref{marker}!    & Référence à un label & \code!\pageref{marker}! & Référence à une page \\
    \code!\footnote{text}! & Note de bas de page  & \code!\url{heig-vd.ch}! & Hyperlien            \\
\end{tabular}
\end{multicols*}

\end{document}
