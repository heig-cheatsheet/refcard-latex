\documentclass{article}

\usepackage[a4paper, landscape, margin=1cm]{geometry}
\usepackage{fontspec}
\usepackage[dvipsnames]{xcolor}
\usepackage[french]{babel}
\usepackage[fontsize=6.5pt]{scrextend}
\usepackage[T1]{fontenc}
\usepackage{multicol}
\usepackage{tabularx}
\usepackage{sectsty}
\usepackage{lmodern}
\usepackage{stix}
\usepackage{listings}
\usepackage{multirow}
\usepackage{float}
\usepackage{titlesec}
\usepackage{fancyvrb,cprotect}
\usepackage{graphicx}
\usepackage{fontspec}
\usepackage{booktabs}
\usepackage{amsmath}
\input{revision}

\ifdefined\tinted
  \pagecolor{SpringGreen!60}
\fi

\setlength{\parskip}{0.2em}
\setlength{\parindent}{0em}

% Highlight configuration for C programming language
\lstset{
  language=tex,
  breaklines=true,
  keywordstyle=\bfseries\color{black},
  basicstyle=\ttfamily\color{black},
  emphstyle={\em \color{gray}},
  commentstyle=\color{gray}\ttfamily,
  mathescape=true,
  keepspaces=true,
  showspaces=false,
  showtabs=true,
  tabsize=3,
  columns=fullflexible,
  aboveskip=0pt,
  belowskip=0pt
}

% Configuration
\renewcommand{\familydefault}{\sfdefault}

\allsectionsfont{\sffamily}

% No pages numbering
\pagenumbering{gobble}

\newlength\mybaselinestretch
\mybaselinestretch=0pt plus 0.02pt\relax
\addtolength{\baselineskip}{\mybaselinestretch}

\setlength\parindent{0pt}
\setlength\tabcolsep{1.5pt}
\setlength{\columnseprule}{0.4pt}

% Macros
\newcommand{\tab}{\hspace{2em}}
\newcommand{\etc}{\small \ldots}
\newcommand{\any}{$\hzigzag$~}
\newcommand{\spc}{$\mathvisiblespace$}
\newcommand{\cd}{\lstinline}

% Titles and paragraphs more compact
% \titlespacing*{\section}{0pt}{1em}{0pt}
% \titlespacing*{\subsection}{0pt}{1em}{0pt}


\begin{document}

\begin{multicols*}{3}
\begin{tabularx}{\columnwidth}{lX}
    \raisebox{-\totalheight}{\includegraphics[width=1cm]{assets/heig-vd-black.pdf}} &
\begin{center}
  {\Large \bf Carte de référence \LaTeX} \\
  version \revision \ -- \revisiondate \\
\end{center}
\end{tabularx}
{
\scriptsize
Cette carte de référence est destinée aux étudiants utilisant \LaTeX2e pour leurs rapports de travaux pratiques ou rapports de travail de Bachelor. Cette carte comprend une liste non exhaustive des possibilités du langage \LaTeX résumé en une feuille A4 recto-verso. Ce document est en partie inspiré de \emph{Learn X in Y}.

}
\section*{Classes de document}
\begin{lstlisting}
\documentclass[11pt,a4paper,landscape,twoside]{article}
\documentclass{report} % Dispose de \chapter
\documentclass{book}   % Dispose de \frontmatter \mainmatter et \backmatter

\documentclass{memoir} % Pour écrire une thèse de doctorat
\documentclass{beamer} % Pour faire des présentation type PPT
\documentclass{exam}   % Pour écrire des travaux écrits/examens
\documentclass{lettre} % Modèle de l'observatoire de Genève pour écrire des lettres.
\end{lstlisting}

\section*{Paquets}

\begin{lstlisting}
\usepackage{caption}
\usepackage{amsmath}
\usepackage{amsfonts}
\usepackage{float}
\usepackage{graphicx}
\usepackage{longtable}
\usepackage{booktabs}
\usepackage{tabularx}
\usepackage{listings}
\usepackage{xcolor}
\usepackage{xifthen}
\usepackage{hyperref}
\usepackage{fancyhdr}
\usepackage{bookmark}
\usepackage{geometry}
\usepackage{setspace}
\usepackage{babel}
\end{lstlisting}

\section*{Formats}
\begin{lstlisting}

\chapter[Description courte]{Description longue}
\section{Section 1}
\subsection{Section 1.1}
\subsubsection{Section 1.1.1}
\paragraph{Section 1.1.1.1}
\label{subsec:foobar}
Comme vu au paragraphe \ref{subsec:foobar}

\textbf{Texte en gras}
\textit{Texte en italique}
\texttt{Texte en couleur}
\textsc{Texte en souligné}
\textup{Texte en exposant}
\textsl{Texte en italique souligné}
\textsf{Texte en gras souligné}

\begin{enumerate} % Liste numérotée
    \item Salade
    \item Melon
    \item Durian
\end{enumerate}

\begin{itemize} % Liste à puces
    \item Un
    \item Deux
    \item Trois
\end{itemize}
\end{lstlisting}
\section*{Minimal}

\begin{lstlisting}
\documentclass[a4paper,11pt]{article}
\usepackage[french]{babel}

\author{Yves Chevallier}
\date{\today}
\title{Apprendre \LaTeX{} avec plaisir}
\begin{document}
\maketitle
\newpage
\tableofcontents
\section{Introduction}
Bonjour.
\end{lstlisting}

\section*{Math}

\begin{tabularx}{\columnwidth}{Xl}
    \lstinline{a^2 + b^2 = c^2} & $a^2 + b^2 = c^2$ \\
\end{tabularx}

\begin{multicols*}{2}

    \begin{tabularx}{\columnwidth}{lX}
        $\alpha A$ & \lstinline{\\alpha A} \\
        $\beta B$ & \lstinline{\\beta B} \\
        $\gamma \Gamma$ & \lstinline{\\gamma \\Gamma} \\
        $\delta \Delta$ & \lstinline{\\delta \\Delta} \\
        $\epsilon \varepsilon E$ & \lstinline{\\epsilon \\varepsilon E} \\
        $\zeta Z Z$ & \lstinline{\\zeta Z} \\
        $\eta H$ & \lstinline{\\eta H} \\
        $\theta \vartheta \Theta$ & \lstinline{\\theta \\vartheta \\Theta} \\
        $\iota I$ & \lstinline{\\iota I} \\
        $\kappa K$ & \lstinline{\\kappa K} \\
        $\lambda \Lambda$ & \lstinline{\\lambda \\Lambda} \\
        $\mu M$ & \lstinline{\\mu M} \\
        $\nu N$ & \lstinline{\\nu N} \\
        $\xi \Xi$ & \lstinline{\\xi \\Xi} \\
        $o O$ & \lstinline{o O} \\
        $\pi \Pi$ & \lstinline{\\pi \\Pi} \\
        $\rho \varrho P$ & \lstinline{\\rho \\varrho P} \\
        $\sigma \Sigma$ & \lstinline{\\sigma \\Sigma} \\
        $\tau T$ & \lstinline{\\tau T} \\
        $\upsilon \Upsilon$ & \lstinline{\\upsilon \\Upsilon} \\
        $\phi \varphi \Phi$ & \lstinline{\\phi \\varphi \\Phi} \\
        $\chi X$ & \lstinline{\\chi X} \\
    $\psi \Psi$ & \lstinline{\\psi \\Psi} \\
    $\omega \Omega$ & \lstinline{\\omega \\Omega} \\
\end{tabularx}
\columnbreak
\begin{tabularx}{\columnwidth}{lX}
    $\leftarrow$ & \lstinline{\\leftarrow} \\
    $\rightarrow$ & \lstinline{\\rightarrow} \\
    $\Leftarrow$ & \lstinline{\\Leftarrow} \\
    $\Rightarrow$ & \lstinline{\\Rightarrow} \\
    $\infty$ & \lstinline{\\infty} \\
    $\forall$ & \lstinline{\\forall} \\
    $\Re$ & \lstinline{\\Re} \\
    $\Im$ & \lstinline{\\Im} \\
    $\nabla$ & \lstinline{\\nabla} \\
    $\exists$ & \lstinline{\\exists} \\
    $\nexists$ & \lstinline{\\nexists} \\
    $\partial$ & \lstinline{\\partial} \\
    $\triangle$ & \lstinline{\\triangle} \\
    $\blacksquare$ & \lstinline{\\blacksquare} \\
    $\times$ & \lstinline{\\times} \\
    $\cdot$ & \lstinline{\\cdot} \\
    $\leq$ & \lstinline{\\leq} \\
    $\geq$ & \lstinline{\\geq} \\
    $\in$ & \lstinline{\\in} \\
    $\notin$ & \lstinline{\\notin} \\
    $\vee$ & \lstinline{\\vee} \\
\end{tabularx}

\end{multicols*}

\section*{Equations}

\begin{lstlisting}
\begin{equation}
    c^2 = a^2 + b^2.
    \label{eq:pythagoras} % for referencing
\end{equation}
\begin{equation}
  \int_{0}^{\infty} \mathrm{e}^{-x} \mathrm{d}x
\end{equation}

\end{lstlisting}

\section*{Figures}

\begin{multicols*}{2}


\begin{lstlisting}
\begin{figure}[H]
  \centering
  \includegraphics[width=3cm]{example-image-duck}
  \caption[Canard]{Un canard}
  \label{fig:right-triangle}
\end{figure}
\end{lstlisting}

\begin{figure}[H]
    \centering
    \includegraphics[width=3cm]{example-image-duck}
    \caption[Canard]{Un canard}
    \label{fig:right-triangle}
\end{figure}

\end{multicols*}

\section*{Table}

\begin{multicols*}{2}


\begin{lstlisting}
\begin{table}[H]
  \centering
  \caption{Caption for the Table.}
  \begin{tabular}{c|cc}
    Number &  Last Name & First Name \\
    1 & Biggus & Dickus \\
    2 & Monty & Python
  \end{tabular}
\end{table}
\end{lstlisting}
\columnbreak
\begin{table}[H]
  \centering
  \caption{Caption for the Table.}
  \begin{tabular}{c|cc}
    \toprule
    Number &  Last Name & First Name \\ \midrule
    1 & Biggus & Dickus \\
    2 & Monty & Python \\ \bottomrule
  \end{tabular}
\end{table}

\end{multicols*}
\end{multicols*}

\end{document}
